\documentclass[10pt]{article}

\usepackage[T2A]{fontenc}
\usepackage[utf8x]{inputenc}

\usepackage[english,russian]{babel}
\usepackage{graphics, graphicx}

\usepackage{amsmath}
\usepackage{amssymb}
\usepackage{amsfonts}


\usepackage[left=20mm, top=20mm, right=20mm, bottom=20mm, nohead, nofoot, footskip=15pt]{geometry}

\usepackage{color}
\usepackage{epsfig}
\usepackage{bm}
\usepackage[colorlinks,urlcolor=blue]{hyperref}
\usepackage{tikz}
\usepackage{pgfplots}



\tolerance=1
\emergencystretch=\maxdimen
\hyphenpenalty=10000
\hbadness=10000


\author{Дмитриев Леонид, группа 317}

\title{
	Обработка и распознавание изображений\\
	Лабораторная работа № 2\\
	Изучение и освоение методов классификации формы изображений
}

\begin{document}
	{
		\LARGE
		\maketitle
	}
	
	\clearpage
	
	
	{ \large \tableofcontents} 
	
	\clearpage
	
	
	
	\section*{Постановка задачи}
	\addcontentsline{toc}{section}{Постановка задачи}
	
	
	Целью работы является разработка и реализация программы для классификации изображений моделей графов,
	построенных из магнитной головоломки.\\
	Программа должна обеспечивать:
	\begin{itemize}
		\item Ввод и отображение на экране изображений в формате JPEG
		
		\item Сегментацию изображений на основе точечных и пространственных преобразований
		
		\item Генерацию признаковых описаний структуры графов на изображении
		
		\item Классификацию изображений по 4 заданным классам
	\end{itemize}
	
	
	\section*{Описание данных}
	\addcontentsline{toc}{section}{Описание данных}
	
	
	
	
	\section*{Описание метода решения}
	\addcontentsline{toc}{section}{Описание метода решения}
	
	
	
	\section*{Описание программной реализации}
	\addcontentsline{toc}{section}{Описание программной реализации}
	
	
	
	
	\section*{Эксперименты}
	\addcontentsline{toc}{section}{Эксперименты}
	
	
	
	
	\section*{Выводы}
	\addcontentsline{toc}{section}{Выводы}
	
	
	
\end{document}
